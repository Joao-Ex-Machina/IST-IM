\documentclass[a4paper,12pt]{article}
\usepackage{graphicx}
\usepackage[a4paper, total={6in, 9in}]{geometry}
\usepackage[T1]{fontenc}
\usepackage[utf8]{inputenc}
\usepackage{graphicx}
\usepackage{tikz}
\usepackage{float}
\usepackage{mathtools}
\usepackage{fancyhdr}
\usepackage{caption}
\usepackage{textgreek}
\usepackage{yfonts}
\graphicspath{/home/Desktop/IST/2ano/2semestre/1quarter/IM/IST-IM/images}
\renewcommand{\figurename}{Figura}
\renewcommand{\contentsname}{Índice}
\renewcommand{\refname}{Bibliografia}
\renewcommand\paragraph{\@startsection{paragraph}{4}{\z@}
% display heading, like subsubsection
                                     {-3.25ex\@plus -1ex \@minus -.2ex}
                                     {1.5ex \@plus .2ex}
                                     {\normalfont\normalsize\bfseries}}
\setcounter{secnumdepth}{4}
\pagestyle{fancy}
\date{Março 2022}
\title{Instrumentação e Medidas \\ \large {Resumo Teórico}}
\author{
\\ João Barreiros C. Rodrigues nº99968
}

\begin{document}
	\pagenumbering{gobble}
	\begin{titlepage} % Suppresses displaying the page number on the title page and the subsequent page counts as page 1
        \newcommand{\HRule}{\rule{\linewidth}{0.5mm}} % Defines a new command for horizontal lines, change thickness here
        \center % Centre everything on the page
        \textsc{\LARGE Instituto Superior Técnico}\\[1.5cm] % Main heading such as the name of your university/college
	\textsc{\Large Licenciatura em Engenharia Eletrotécnica e de Computadores}\\[0.25cm]
        \textsc{\Large Intrumentação e Medidas}\\[0.5cm] % Major heading such as course name
        \HRule\\[0.4cm]
        {\huge\bfseries Resumo Teórico}\\[0.4cm] % Title of your document
        \HRule\\[1.5cm]\
        João \textsc{Barreiros C. Rodrigues} nº99968 , LEEC\\
	\vfill\vfill\vfill % Position the date 3/4 down the remaining page
        {\large Março 2022} % Date, change the \today to a set date if you want to be precise
        \vfill % Push the date up 1/4 of the remaining page
\end{titlepage}
	\pagenumbering{arabic}
	\newpage
		\tableofcontents
	\newpage
	\section{Noções Introdutórias da Instrumentação e Medidas}
		\subsection{Distinção entre Exatidão e precisão}
			\par
			\textbf{Exatidão:} Segundo o artigo 2.13 do VIM,descreve a proximidade dos valores medidos com o valor real \par
			\textbf{Precisão:} Segundo o artigo 2.15 do VIM,descreve o grau de concordância entre valores medidos\par

			\begin{figure}[h]
				\centering
					\captionsetup{justification=centering}
					\includegraphics[width=0.5\textwidth]{01.png}
					\caption{Analogia com tiro com arco.}
			\end{figure}
		\subsection{Metodologia de uma medição (de uma grandeza eléctrica)}
			\par
			\textbf{1º} Medição directa \par
			\textbf{2º} Alcance \par
			\textbf{3º} Cálculo do erro de calibração com base na data de aquisição do equipamento de medida
		\subsection{Propagação de Incertezas}
			Para o cálculo de propagação da incerteza existem 3 métodos principais.
			\subsubsection{Lei da Propagação da Incerteza (LPI)}

			\subsubsection{Propagação Analítica de PDF's}
			\subsubsection{Método de Monte Carlo}
	\section{Sinais Elétricos}
		\subsection{Valor Eficaz}
		            Valor de Tensão contínua que produz a mesma libertação de energia por Efeito Joule que a resistência.
			\subsubsection{Método RMS (Root Min Square)}\par
			

			\subsubsection{Distorção harmónica}
			\subsubsection{Factor de rejeiçao}
			\subsubsection{Exemplos com Ondas Triviais}
	\section{Teoria de AmpOp's}
		\subsection{Montagens base e suas características}
		\subsection{Ganho}
	\section{Conversores}



\end{document}
